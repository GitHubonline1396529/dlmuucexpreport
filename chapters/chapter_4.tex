\chapter{实验步骤与结果}

\section{文件目录说明}

项目的主文件夹下的各个目录及其说明如下列所示:

\begin{itemize}
    \item \texttt{figure}文件夹:用于储存报告中的各项图;
    \item \texttt{chapters}文件夹:分章节保存报告主题内容,\texttt{chapters\_1.tex}到\texttt{chapter\_6.tex}分别是章节1到6的内容。初始状态下为\texttt{example.pdf}文件中的示例文本内容;
    \item \texttt{code}文件夹用于保存需要插入报告的代码文件。
    \item \texttt{fonts}用于保存符号报告样式的字体文件。
    \item \texttt{document}文件夹保存了一份报告排版格式的要求文件。
    \item \texttt{backup}文件夹保存了一份GB7714-2015标准引用格式的副本文件,以防万一\texttt{biblatex}自动调用引用格式不生效的情况下可以手动调用。
\end{itemize}

\section{获取模板}

本模板的GitHub开源地址位于\href{https://github.com/GitHubonline1396529/dlmuucexpreport}{GitHubonline1396529/dlmuucexpreport} ,国内镜像地址位于Gitee上的\href{https://gitee.com/BOXonline_1396529/dlmuucexpreport}{BOXonline\_1396529/dlmuucexpreport} 。点击链接访问。您可以在网站下载源码的压缩包,或使用Git拉取。

本模板项目是开源模板,旨在免费提供。如果您在任何平台通过收费途径获取到本模板,请当心!您可能受到了欺诈。

\section{编译文档}

\subsection{使用make命令编译文档}

如果您习惯使用\texttt{make}命令,并且您的计算机上已经正确配置了GNU Make(譬如基于mingw环境,或者您在大一学习C语言的时候安装过GCC组件作为您的C语言编译器,或者您安装过Git for Windows),那么文档的编译规则已经在Makefile中正确定义。使用\texttt{make}命令能够即刻生成您的文档。

\begin{itemize}
    \item 如果您需要删除工作区目录下包含成品文档在内的所有生成文件,请使用\texttt{make clean}
    \item 如果您只是需要删除所有的辅助文件,请使用\texttt{make clear}
    \item \texttt{makefile}中定义的\texttt{make example}命令是用于构建您当前看到的这份示例文档的,通常情况下您不会使用到。
\end{itemize}

\subsection{使用\XeLaTeX 编译文档}

文档需按照如下顺序编译:

XeLaTeX → Biber → XeLaTeX → XeLaTeX

通过在命令行执行如代码\ref{lst:shell_script}所示的命令,你可以正确编译文档:

\begin{lstlisting}[
    language=bash,
    caption={用于编译文档的shell命令}\label{lst:shell_script}]
xelatex main.tex
biber main # 编译引用格式
xelatex main.tex
xelatex main.tex
latexmk -c # 清理工作区
\end{lstlisting}

\subsection{使用\LaTeX{}mk基于\XeLaTeX 构建文档}

您也可以通过\LaTeX{}mk指定使用\XeLaTeX 规则编译文档。由于新版本增加了\texttt{latexmkrc}配置文件,您在使用\LaTeX{}mk时无需额外指定\texttt{-xelatex}参数。编译的命令如\ref{lst:shell_cmd}所示。

\begin{lstlisting}[
    language=bash,
    caption={用于编译文档的LaTexmk命令}\label{lst:shell_cmd}]
latexmk main.tex
\end{lstlisting}

\section{写作示例}

\textbf{以下的写作示例来自\href{https://github.com/ElegantLaTeX/ElegantNote/blob/master/elegantnote-cn.tex}{ElegantNote}项目},此处仅作展示使用:我们将通过三个步骤定义可测函数的积分。首先定义非负简单函数的积分。以下设 $E$ 是 $\mathcal{R}^n$ 中的可测集。

\begin{definition}[可积性]
设 $ f(x)=\sum\limits_{i=1}^{k} a_i \chi_{A_i}(x)$ 是 $E$ 上的非负简单函数,其中 $\{A_1,A_2,\ldots$, $A_k\}$ 是 $E$ 上的一个可测分割,$a_1,a_2,\ldots,a_k$ 是非负实数。定义 $f$ 在 $E$ 上的积分为 1. 3
\begin{equation}
   \label{inter}
   \int_{E} f dx = \sum_{i=1}^k a_i m(A_i).
\end{equation}
一般情况下 $0 \leq \int_{E} f dx \leq \infty$。若 $\int_{E} f dx < \infty$,则称 $f$ 在 $E$ 上可积。
\end{definition}

一个自然的问题是,Lebesgue 积分与我们所熟悉的 Riemann 积分有什么联系和区别?之后我们将详细讨论 Riemann 积分与 Lebesgue 积分的关系。这里只看一个简单的例子。设 $D(x)$ 是区间 $[0,1]$ 上的 Dirichlet 函数。即 $D(x)=\chi_{Q_0}(x)$,其中 $Q_0$ 表示 $[0,1]$ 中的有理数的全体。根据非负简单函数积分的定义,$D(x)$ 在 $[0,1]$ 上的 Lebesgue 积分为
\begin{equation}\label{inter2}
  \int_0^1 D(x)dx = \int_0^1 \chi_{Q_0} (x) dx = m(Q_0) = 0
\end{equation}
即 $D(x)$ 在 $[0,1]$ 上是 Lebesgue 可积的并且积分值为零。但 $D(x)$ 在 $[0,1]$ 上不是 Riemann 可积的。

\begin{table}[htbp]
  \centering
  \small
  \caption{燃油效率与汽车价格}
    \begin{tabular}{lcc}
    \toprule
                  &       (1)         &        (2)      \\
    \midrule
    燃油效率      &   -238.90***      &      -49.51     \\
                  &    (53.08)        &      (86.16)    \\
    汽车重量      &                   &        1.75***  \\
                  &                   &       (0.641)   \\
    常数项        &  11253.00***      &    1946.00      \\
                  &  (1171.00)        &   (3597.00)     \\
    观测数        &     74            &      74         \\
    $R^2$         &      0.220        &       0.293     \\
    \bottomrule
    \end{tabular}%
  \label{tab:reg}%
\end{table}%

\begin{theorem}[Fubini 定理]\label{thm:fubi}
若 $f(x,y)$ 是 $\mathcal{R}^p\times\mathcal{R}^q$ 上的非负可测函数,则对几乎处处的 $x\in \mathcal{R}^p$,$f(x,y)$ 作为 $y$ 的函数是 $\mathcal{R}^q$ 上的非负可测函数,$g(x)=\int_{\mathcal{R}^q}f(x,y) dy$ 是 $\mathcal{R}^p$ 上的非负可测函数。并且
\begin{equation}\label{eq:461}
  \int_{\mathcal{R}^p\times\mathcal{R}^q} f(x,y) dxdy=\int_{\mathcal{R}^p}\left(\int_{\mathcal{R}^q}f(x,y)dy\right)dx.
\end{equation}
\end{theorem}

\begin{proof}
Let $z$ be some element of $xH \cap yH$.  Then $z = xa$ for some $a \in H$, and $z = yb$ for some $b \in H$. If $h$ is any element of $H$ then $ah \in H$ and $a^{-1}h \in H$, since $H$ is a subgroup of $G$. But $zh = x(ah)$ and $xh = z(a^{-1}h)$ for all $h \in H$. Therefore $zH \subset xH$ and $xH \subset zH$, and thus $xH = zH$.  Similarly $yH = zH$, and thus $xH = yH$, as required.
\end{proof}

回归分析 (regression analysis) 是确定两种或两种以上变量间相互依赖的定量关系的一种统计分析方法。根据定理~\ref{thm:fubi},其运用十分广泛,回归分析按照涉及的变量的多少,分为一元回归和多元回归分析;按照因变量的多少,可分为简单回归分析和多重回归分析;按照自变量和因变量之间的关系类型,可分为线性回归分析和非线性回归分析。

