\chapter{实验步骤与结果}

\section{文件目录说明}

项目的主文件夹下的各个目录及其说明如下列所示:

\begin{itemize}
    \item \texttt{figure}文件夹:用于储存报告中的各项图;
    \item \texttt{chapters}文件夹:分章节保存报告主题内容,\texttt{chapters\_1.tex}到\texttt{chapter\_6.tex}分别是章节1到6的内容。初始状态下为\texttt{example.pdf}文件中的示例文本内容;
    \item \texttt{code}文件夹用于保存需要插入报告的代码文件。
    \item \texttt{fonts}用于保存符号报告样式的字体文件。
    \item \texttt{document}文件夹保存了一份报告排版格式的要求文件。
    \item \texttt{backup}文件夹保存了一份GB7714-2015标准引用格式的副本文件,以防万一\texttt{biblatex}自动调用引用格式不生效的情况下可以手动调用。
\end{itemize}

\section{获取模板}

本模板的GitHub开源地址位于\href{https://github.com/GitHubonline1396529/dlmuucexpreport}{GitHubonline1396529/dlmuucexpreport} ,国内镜像地址位于Gitee上的\href{https://gitee.com/BOXonline_1396529/dlmuucexpreport}{BOXonline\_1396529/dlmuucexpreport} 。点击链接访问。您可以在网站下载源码的压缩包,或使用Git拉取。

本模板项目是开源模板,旨在免费提供。如果您在任何平台通过收费途径获取到本模板,请当心!您可能受到了欺诈。

\section{编译文档}

\subsection{使用make命令编译文档}

如果您习惯使用\texttt{make}命令,并且您的计算机上已经正确配置了GNU Make(譬如基于mingw环境,或者您在大一学习C语言的时候安装过GCC组件作为您的C语言编译器,或者您安装过Git for Windows),那么文档的编译规则已经在Makefile中正确定义。使用\texttt{make}命令能够即刻生成您的文档。

\begin{itemize}
    \item 如果您需要删除工作区目录下包含成品文档在内的所有生成文件,请使用\texttt{make clean}
    \item 如果您只是需要删除所有的辅助文件,请使用\texttt{make clear}
    \item \texttt{makefile}中定义的\texttt{make example}命令是用于构建您当前看到的这份示例文档的,通常情况下您不会使用到。
\end{itemize}

\subsection{使用\XeLaTeX 编译文档}

文档需按照如下顺序编译:

XeLaTeX → Biber → XeLaTeX → XeLaTeX

通过在命令行执行如代码\ref{lst:shell_script}所示的命令,你可以正确编译文档:

\begin{lstlisting}[
    language=bash,
    caption={用于编译文档的shell命令}\label{lst:shell_script}]
xelatex main.tex
biber main # 编译引用格式
xelatex main.tex
xelatex main.tex
latexmk -c # 清理工作区
\end{lstlisting}

\subsection{使用LaTexmk基于\XeLaTeX 构建文档}

您也可以通过LaTexmk指定使用\XeLaTeX 规则编译文档,这只需要您在使用LaTexmk时指定\texttt{-xelatex}参数:

\begin{lstlisting}[
    language=bash,
    caption={用于编译文档的LaTexmk命令}\label{lst:shell_script}]
latexmk -xelatex main.tex
\end{lstlisting}
