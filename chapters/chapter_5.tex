\chapter{实验总结与心得体会}

\section{帮助完善这篇模板}\label{sec:joinus}

如果您在模板中发现任何不足,欢迎参与模板的改进工作。您可以:

\begin{itemize}
    \item 在本模板的\href{https://github.com/GitHubonline1396529/dlmuucexpreport}{GitHub Repo}提交相应的\href{https://github.com/GitHubonline1396529/dlmuucexpreport/issues}{Issue}/\href{https://github.com/GitHubonline1396529/dlmuucexpreport/pulls}{Pull Request}
    \item 创建模板的Fork。
\end{itemize}

如果你喜欢这个模板,别忘了帮忙点亮Git仓库的小星星。

\section{更新日志}

\begin{itemize}
    \item 2024年7月23日:允许修改封面上的标题和学校名称
    \item 2024年8月1日:修正了章节标题换行的错误;变更文档类型由ctexart为ctexbook
    \item 2024年8月21日:补充说明了有关封面样式差异的问题
    \item 2024年9月13日:修复了封面样式的细微差异;修正了引用文献角标没有方括号的问题。
    \item 2024年9月26日:改进表格字号控制;修正了款项标题编号不显示的问题,改进了款项标题样式。
    \item 2024年10月10日:修正了款项标题的段前段后间距为0行
    \item 2024年10月14日:在文档中新增了文件目录说明,将封面元数据设置迁移至导言区
    \item 2024年10月26日:覆盖了\texttt{maketitle}命令使文档类设计更规范;文档类名称变更
    \item 2024年10月27日:更新了 Makefile 清理辅助文件的逻辑;变更文档类继承为 `report`;兼容 `abstract` 宏包以便用户使用摘要。
    \item 2025年3月14日:将生成封面的逻辑固定到文件\texttt{thecover.sty};在生成的PDF文件的书签目录中显示章节编号;。设置 URL 链接为蓝色;新增若干个 Theorem 环境。
    \item 2025年3月17日:补充文档注释信息。
    \item 2025年3月19日:增加了\texttt{latexmkrc}以强化构建系统,同时相当于为Overleaf提供了支持。
\end{itemize}
