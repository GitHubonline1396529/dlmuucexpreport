\section{实验步骤与结果}

\subsection{编译文档}

\subsubsection{使用make命令编译文档}

如果您习惯使用\texttt{make}命令,并且您的计算机上已经正确配置了GNU Make(譬如基于mingw环境,或者您在大一学习C语言的时候安装过GCC组件作为您的C语言编译器,或者您安装过Git for Windows),那么文档的编译规则已经在Makefile中正确定义。使用\texttt{make}命令能够即刻生成您的文档。

\subsubsection{使用\XeLaTeX 编译文档}

文档需按照如下顺序编译:

XeLaTeX → Biber → XeLaTeX → XeLaTeX

通过在命令行执行如代码\ref{lst:shell_script}所示的命令,你可以正确编译文档:

\begin{lstlisting}[
    language=bash,
    caption={用于编译文档的shell命令}\label{lst:shell_script}]
xelatex main.tex
biber main # 编译引用格式
xelatex main.tex
xelatex main.tex
latexmk -c # 清理工作区
\end{lstlisting}